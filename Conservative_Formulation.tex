\section{Simple Conservative Formulation} \label{Conservative}
This section introduces our first novel approximation of the robust program in \eqref{GP_counterparts_finite}. While this formulation is the simplest to implement, it is also the most conservative and completely decouples the monomials of each posynomial.

\theoremstyle{definition}
\begin{definition}
A solution $\vec{x}$ to a geometric program is said to be feasible only if $\vec{x}$ satisfies all constraints of that geometric program.
\end{definition}

\begin{definition}
Let $\mathcal{C}$ be a constraint on $\vec{x}$, and let $\mathcal{S}$ be a set of constraints on $\vec{x}$ and some additional variables $\vec{y}$. $\mathcal{S}$ is said to be a safe approximation for $\mathcal{C}$ if the $\vec{x}$ component of every feasible solution $\left(\vec{x},\vec{y}\right)$ is $\mathcal{S}$-feasible to $\mathcal{C}$ \cite{ben-tal_ghaoui_nemirovski_2009}.
\end{definition}
For example, if $f\left(x\right) \leq g\left(x\right)$ $\forall x$, then $g\left(x\right) \leq 1$ is a safe approximation of $f\left(x\right) \leq 1$.

One way to approach the intractability of the dual-form problems in \eqref{GP_counterparts_finite} is to replace each constsraint with a tractable safe approximation. The fact that
$$\max_{\vec{\zeta} \in \mathcal{Z}} \textstyle{\sum}_{k=1}^{K_i}e^{\vec{a_{ik}}\left(\zeta\right)\vec{x} + b_{ik}\left(\zeta\right)} \leq \sum_{k=1}^{K_i} {\displaystyle \max_{\vec{\zeta} \in \mathcal{Z}}}\ e^{\vec{a_{ik}}\left(\zeta\right)\vec{x} + b_{ik}\left(\zeta\right)}$$
suggests replacing the intractable constraints by 
\begin{equation}
	\sum_{k=1}^{K_i}\max_{\vec{\zeta} \in \mathcal{Z}} \left\{e^{\vec{a_{ik}}\left(\zeta\right)\vec{x} + b_{ik}\left(\zeta\right)}\right\} \leq 1 \qquad \forall i \in 1,...,m
	\label{conservative_robust_constraint}
\end{equation}
From \eqref{conservative_robust_constraint}, a suggested safe approximation of  \eqref{GP_counterparts_finite} is
\begin{equation}
\begin{aligned}
& \min &&f_0\left(\vec{x}\right)\\
& \text{subject to} &&\textstyle{\sum}_{k=1}^{K_i} {\displaystyle \max_{\vec{\zeta} \in \mathcal{Z}}} \left\{e^{\vec{a_{ik}}\left(\zeta\right)\vec{x} + b_{ik}\left(\zeta\right)}\right\} &&\leq 1 &&\forall i \in 1,...,m
\end{aligned}
\label{GP_safe_conservative}
\end{equation}
By adding some dummy variables, the above can be rewritten as
\begin{equation}
\begin{aligned}
& \min &&f_0\left(\vec{x}\right)\\
& \text{subject to} &&\textstyle{\sum}_{k=1}^{K_i}e^{t_{ik}} &&\leq 1 &&\forall i \in 1,...,m \\
& &&\max_{\vec{\zeta} \in \mathcal{Z}} \left\{e^{\vec{a_{ik}}\left(\zeta\right)\vec{x} + b_{ik}\left(\zeta\right)}\right\} &&\leq e^{t_{ik}} &&\forall i \in 1,...,m &&\forall k \in 1,...,K_i\\
\end{aligned}
\label{GP_safe_decoupled}
\end{equation}
And finally, in convex form
\begin{equation}
\begin{aligned}
& \min &&\log\left(f_0\left(\vec{x}\right)\right)\\
& \text{subject to} &&\log(\textstyle{\sum}_{k=1}^{K_i}e^{t_{ik}}) &&\leq 0 &&\forall i \in 1,...,m \\
& &&\max_{\vec{\zeta} \in \mathcal{Z}} \left\{\vec{a_{ik}}\left(\zeta\right)\vec{x} + b_{ik}\left(\zeta\right)\right\} &&\leq t_{ik} &&\forall i \in 1,...,m &&\forall k \in 1,...,K_i\\
\end{aligned}
\label{GP_safe_convex}
\end{equation}
Note that dummy variables need not be introduced for constraints with $K_i = 1$.
 
The convex constraints in \eqref{GP_safe_convex} are deprived of data; uncertainty is present only in the linear constraints. As a result, techniques from robust linear programming (Appendix \ref{LP_to_GP}) can be used to robustify the uncertain constraints. Under a polyhedral uncertainty set these robust linear constraints are equivalent to a set of linear constraints, allowing the approximate RGP to be transformed into a GP. A similar transformation can also be done for coefficients in an elliptical uncertainty set.

If the exponents are also in an elliptical uncertainty set, the transformed program is no longer a GP but becomes a cone program with exponential- and second-order-cone (SOC) constraints. The authors have used programs such as \textit{ECOS} \cite{bib:Domahidi2013ecos} and \textit{SCS} \cite{o’donoghue_chu_parikh_boyd_2016} to solve such problems; they can also be solved as a GP by further approximating SOC constraints with GP-compatible softmax functions \cite{hoburg_kirschen_abbeel_2016}.

\begin{definition}
Two monomials are directly dependent if both depend on the same perturbation element $\zeta_i$. e.g. $M_1 = e^{\vec{a_1}(\zeta_1)\vec{x} + b_1}$ and $M_2 = e^{\vec{a_3}(\zeta_1, \zeta_2)\vec{x} + b_3}$ are directly dependent as they share the perturbation element $\zeta_1$.
\end{definition}

\begin{definition}
Two monomials are indirectly dependent if they are both dependent on a third monomial. e.g. with $M_1 = e^{\vec{a_1}(\zeta_1)\vec{x} + b_1}$, $M_2 = e^{\vec{a_1}(\zeta_2)\vec{x} + b_2}$, and $M_3 = e^{\vec{a_3}(\zeta_1, \zeta_2)\vec{x} + b_3}$, $M_1$ and $M_2$ are indirectly dependent because both share a perturbation element with (and are hence directly dependent on) $M_2$.
\end{definition}

\begin{definition}
Two monomials are independent if they are neither directly nor indirectly dependent.
\end{definition}

\begin{definition}
Two coefficients are consistently dependent if they both increase or both decrease as each perturbation element varies. e.g. if $b_{11} = b_{11}^0 + \zeta_1$ and $b_{12} = b_{12}^0 + 3\zeta_1 - \zeta_2$, then $b_{11}$ and $b_{12}$ are dependent in a consistent way because their coefficients for $\zeta_1$ have the same sign.
\end{definition}

Despite the decoupling of monomials, this formulation is exactly equivalent to \eqref{GP_counterparts_finite} if each posynomial satisfies at least one of the following conditions:
\begin{itemize}
	\item $C_1$: Only one monomial in the posynomial has uncertain parameters
	\item $C_2$: All monomials in the posynomial are either independent or dependent by a shared uncertain monomial factor, e.g. $$
	\begin{aligned}
	p &= e^{a_1\left(\zeta\right)\vec{x}_1 + a_2\vec{x}_2 + b_1\left(\zeta\right)} + e^{a_5\vec{x}_1 + b_3} + e^{a_1\left(\zeta\right)\vec{x}_1 + a_3\vec{x}_3 + b_1\left(\zeta\right)} + e^{a_1\left(\zeta\right)\vec{x}_1 + a_4\vec{x}_2 + b_1\left(\zeta\right)}\\
	&= e^{a_1\left(\zeta\right)\vec{x}_1 + b_1\left(\zeta\right)}\left(e^{a_2\vec{x}_2} + e^{a_3\vec{x}_3} + e^{a_4\vec{x}_2}\right) + e^{a_5\vec{x}_1 + b_3}
	\end{aligned}
	$$
	\item $C_3$: The perturbation set is independent, e.g. $\mathcal{Z} = \left\{ \vec{\zeta} \in \mathbb{R}^L: \|\vec{\zeta}\|_{\infty} \leq \Gamma\right\}$. Moreover, the monomials in the posynomial are either independent, or their coeffcients are consistently dependent.
\end{itemize}
\ \\
Specifically, if the $i^{th}$ posynomial satisfies $C_1$, $C_2$, or $C_3$ then its maximum under perturbation will always equal the sum of the maximums of each decoupled monomial:
$$
\max_{\vec{\zeta} \in \mathcal{Z}} \left\{\textstyle{\sum}_{k=1}^{K_i}e^{\vec{a_{ik}}\left(\zeta\right)\vec{x} + b_{ik}\left(\zeta\right)}\right\} = \sum_{k=1}^{K_i}\max_{\vec{\zeta} \in \mathcal{Z}} \left\{e^{\vec{a_{ik}}\left(\zeta\right)\vec{x} + b_{ik}\left(\zeta\right)}\right\}
$$

Even when applied to a geometric program whose every posynomial does not satisfy $C_1$, $C_2$, or $C_3$, this formulation has the advantage of requiring only a small number of GP or conic constraints. If $\mathbf{P} = \left\{i:K_i >=3\right\}$, $\mathbf{N} = \left\{i:K_i = 2\right\}$, and $\mathbf{M} = \left\{i:K_i = 1\right\}$ then the total number of constraints in \eqref{GP_safe_convex} is 
$$\textstyle{\sum}_{k \in \mathbf{P}} (K_i+1) + 3|\mathbf{N}| + |\mathbf{M}|$$
which is 
$$(r-1)\textstyle{\sum}_{k \in \mathbf{P}} (K_i-3) + 2(r-2)|\mathbf{P}| + (r-3)|\mathbf{N}|$$
constraints fewer than in the two-term formulation of \cite{hsiung_kim_boyd_2007}.
Specifically, this decoupled-monomial formulation will always have fewer constraints than \eqref{Boyd_formulation} if 3 or more piecewise sections are used, which will almost always be the case; three linear sections form a poor approximation of a two-term posynomial. Furthermore, this formulation is only more conservative than that in \eqref{Boyd_formulation} if the number of sections is high enough and one of the approximated two-term posynomials captures an active dependence.