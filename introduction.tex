\section{Introduction} \label{intro}
A geometric program is a log-convex optimization problem of the form
\begin{equation}
\begin{aligned}
	& \text{minimize} && f_0 \left(\vec{u}\right) \\
	& \text{subject to} && f_i \left(\vec{u}\right) \leq 1, i = 1,...,m_p\\
	& && h_i \left(\vec{u}\right) = 1, i = 1, ...,m_e\\
\end{aligned}
\label{GP_standard}
\end{equation}
where each $f_i$ is a {\em posynomial} and each $h_i$ is a {\em monomial}. In its \textbf{natural form}, a monomial $h(\vec{u})$ is a function of the form
\begin{displaymath}
	h(\vec{u}) = e^{b}\textstyle{\prod}_{j=1}^{n}{u_j}^{a_j}
\end{displaymath}
where $\vec{a}$ is a row vector in $\mathbb{R}^n$, $\vec{u}$ is a column vector in $\mathbb{R}^n_+$ , and $b$ is in $\mathbb{R}$. A posynomial $f(\vec{u})$ is the sum of $K \in \mathbb{Z}^+$ monomials
\begin{displaymath}
	f(\vec{u}) = \textstyle{\sum_{k=1}^{K}}e^{b_{kj}}\prod_{j=1}^{n}{u_j}^{a_{kj}}
\end{displaymath}
where $\vec{a_{k}}$ is a row vector in $\mathbb{R}^n$, $\vec{u}$ is a column vector in $\mathbb{R}^n_+$, and $b_{k}$ is in $\mathbb{R}$ \cite{GP_tutorial}.\\
A logarithmic change of the variables $x_j = \log(u_j)$ would convert the monomials and posynomials into their \textbf{exponential form}. A monomial $h_i(\vec{x})$ is a function of the form
\begin{displaymath}
    h_i(\vec{x}) = e^{\vec{a_i}\vec{x} + b_i}
\end{displaymath}
where $\vec{a_i}$ is a row vector in $\mathbb{R}^n$, $\vec{x}$ is a column vector in $\mathbb{R}^n$ , and $b_i$ is in $\mathbb{R}$. A posynomial $f_i(\vec{x})$ is the sum of $K_i \in \mathbb{Z}^+$ monomials
\begin{displaymath}
    f_i(\vec{x}) = \textstyle{\sum_{k=1}^{K_i}}e^{\vec{a_{ik}}\vec{x} + b_{ik}}
\end{displaymath}
where $\vec{a_{ik}}$ is a row vector in $\mathbb{R}^n$, $\vec{x}$ is a column vector in $\mathbb{R}^n$, and $b_{ik}$ is in $\mathbb{R}$.

Solving large-scale GPs became extremely efficient and more reliable \cite{GP_tutorial} after the development of interior point methods \cite{nesterov_nemirovskii_1994} \cite{kortanek_xu_ye_1997}, prompting engineers to use GP formulations for the modeling and design optimization of complex engineering systems. Recent GP-compatible applications have been found in fields as varied as digital circuit optimization \cite{boyd_kim_patil_horowitz_2005}, analog circuit design \cite{hershenson_2004}, communication systems \cite{chiang_2005}, chemical process control \cite{wall_greening_woolsey_1986}, environmental quality control \cite{greenberg_1995}, statistics \cite{mazumdar_jefferson_1983}, and antenna and aircraft design \cite{babakhani_lavaei_doyle_hajimiri_2010} \cite{hoburg_abbeel_2014}.

While these GPs can be solved to global optimality for fixed parameter values, many of those values are uncertain during the engineering design process, either because of intrinsic uncertainty (e.g. wind velocity), or unpredictability at that stage of the design process (e.g. the final assembled weight of an airplane's avionics). Designers, seeking a solution they can trust enough to manufacture, account for these uncertainties by making their models conservative. This conservativeness almost always takes the form of specifying worst-case values and margins for all uncertain constraints and parameters \cite{hoburg_abbeel_2014}. Such an approach exaggerates the impact of uncertainties, leading to sub-optimality which is undesired. From a designer's perspective, however, this unwanted conservativeness is their only way to ensure robust results. New techniques that can represent design uncertainties less conservatively are thus able to directly improve engineering solutions and present new design oppourtunities.

Robust optimization has been used to efficiently solve optimization problems under uncertainty \cite{soyster_1973}. In constrast to stochastic optimization \cite{birge_louveaux_2011} \cite{prékopa_2010}, probability density functions are replaced in robust optimization by uncertainty sets. In the late 1990s, Ben-Tal, Nemirovski, and El Ghaoui proved the tractability of robust linear, quadratic, and second order conic programming problems \cite{ben-tal_ghaoui_nemirovski_2009}. Later, Bertsimas et al. derived a tractable approximation of a subset of robust conic optimization problems \cite{bertsimas_sim_2005}, and discussed the importance and means of choosing an uncertainty set to best describe the problem while preserving tractability \cite{bandi_bertsimas_2012}.

Robust geometric programming is co-NP hard in its natural posynomial form \cite{RGPcoNP}, but an interesting tractable formulation for an approximate robust geometric program was presented by Hsiung, Kim, and Boyd \cite{hsiung_kim_boyd_2007}. They proposed a bivariate safe approximation for the posynomial constraints, replacing each posynomial constraint with a set of two-term (``bivariate'') posynomial constraints that can be easily approximated using piecewise-linear functions. Although it simplifies the problem into a linear optimization problem, the piecewise-linear approximation of every posynomial leads to a considerable number of additional constraints. Moreover, because it decouples all but two monomials in each posynomial, it yields an effective uncertainty set larger than the specified one and thus more conservative. Our methods seek to improve on \cite{hsiung_kim_boyd_2007} by reducing the number of constraints while more accurately approximating the uncertainty set to achieve better (less conservative) solutions.

In this article we use three novel methodologies to construct a tractable approximation of a robust geometric program while drastically reducing the number of constraints compared to \cite{hsiung_kim_boyd_2007}. Our first approach deals with each monomial while separating it from its posynomial. Our second approach assumes that only the coefficients $b$ are uncertain and deals with posynomials as linear constraints. Our third approach accounts for dependency between monomials while letting both the coefficients $b$ and the exponents $\vec{a}$ be uncertain.

In section \ref{RGP} of this article, we define robust geometric programming and derive its robust counterparts. Section \ref{Conservative} introduces a simple formulation, our most conservative proposal, which is then refined in Section \ref{EqIntFor} with the concept of monomial partitioning. Section \ref{twoTerm} briefly reviews the piecewise-linearization of two term posynomials in \cite{hsiung_kim_boyd_2007}, an essential tool for our work. Section \ref{k_term} discusses different methodologies for dealing with larger posynomials. In section \ref{apps} we apply all methodologies, both new and old, to several previously published GPs for aircraft design and discuss the results. Finally, section \ref{conc} states our contributions and conclusions.