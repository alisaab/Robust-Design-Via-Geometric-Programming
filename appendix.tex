\appendix

\section{Robust Linear Programming} \label{LP_to_GP}

This section reviews robust linear programming, a building block to formulate a tractable approximate robust geometric program. Two different examples of perturbation sets will be used for clarification: box and elliptical. Those sets will be used for our discussion throughout the article due to the fact that they are important and frequently used.

Consider the system of linear constraints
\begin{displaymath}
    \mat{A}\vec{x} + \vec{b} \leq 0
\end{displaymath}
where
\begin{equation*}
\begin{aligned}
\mat{A} &\text{ is $m \times n$}\\
\vec{x} &\text{ is $n \times 1$}\\
\vec{b} &\text{ is $m \times 1$}\\
\end{aligned}
\end{equation*}
Assuming that data is uncertain and is given by equations \eqref{Data} and \eqref{perturbation_set}, the constraints should be satisfied for all values of $\vec{\zeta}$, and thus the robust counterparts of the $i^{th}$ linear constraint is
\begin{equation*}
    \sup_{\vec{\zeta} \in \mathcal{Z}}\left\{ \textstyle{\sum}_{l=1}^{L} \zeta_l([\vec{a}^l_i]^T \vec{x} + b^l_i) \right\} \leq - [\vec{a}^0_i]^T\vec{x} - b^0_i
\end{equation*}
This is equivalent to the optimization problem
\begin{equation}
    \max_{\vec{\zeta},\vec{u}} \left\{ \textstyle{\sum}_{l=1}^{L} \zeta_l([\vec{a}^l_i]^T \vec{x} + b^l_i): \mat{F}\vec{\zeta} + \mat{G}\vec{u} + \vec{h} \in \mathbf{K} \right\} \leq - [\vec{a}^0_i]^T\vec{x} - b^0_i
\label{linear_counterparts}
\end{equation}
Applying the conic duality theorem, equation \eqref{linear_counterparts} is equivalent to
\begin{equation}
\begin{aligned}
\vec{h}^T\vec{y}_i + [\vec{a}_i^0]^T\vec{x} + b_i^0 &\leq 0\\
\mat{G}^T\vec{y}_i &= \vec{0}\\
(\mat{F}^T\vec{y}_i)_l + [\vec{a}_i^l]^T\vec{x} + b_i^l &= 0 \quad l = 1,2,...,L\\
\vec{y}_i &\in \mathbf{K}^*
\end{aligned}
\label{robust_linear_general}
\end{equation}
where $\vec{y}_i \in \mathbb{R}^N$, and $\mathbf{K}^*$ is the dual cone of $\mathbf{K}$ (see Appendix \ref{cones}) \cite{ben-tal_ghaoui_nemirovski_2009}.

\subsection{Box Uncertainty Sets} \label{box_linear_app}
If the perturbation set $\mathcal{Z}$ given in equation \eqref{perturbation_set} is a box uncertainty set, i.e. $\|\vec{\zeta}\|_{\infty} \leq \Gamma$, then 
\begin{itemize}
	\item $\mat{F}\vec{\zeta} = [\vec{\zeta};0]$ 
	\item $\mat{G} = \mat{0}$, $\vec{h} = [\vec{0}_{L \times 1};\Gamma]$
	\item $\mathbf{K} = \left\{(\vec{z};t) \in \mathbb{R}^L \times \mathbb{R}: t > \|\vec{z}\|_{\infty} \right\}$ 
	\item the dual cone $\mathbf{K}^* = \left\{(\vec{z};t) \in \mathbb{R}^L \times \mathbb{R}: t > \|\vec{z}\|_{1} \right\}$
\end{itemize}
and therefore, equation \eqref{linear_counterparts} is equivalent to
\begin{equation}
\Gamma \textstyle{\sum}_{l=1}^L |- {b}^l_{i} - \vec{a}^l_i\vec{x}| + \vec{a}^0_i\vec{x} + b^0_i \leq 0
\label{box_absolute}
\end{equation}
If only $\vec{b}$ is uncertain, i.e. $\mat{A}^l = 0 \quad \forall l = 1,2,...,L$, then equation \eqref{box_absolute} will become
\begin{equation}
\textstyle{\sum}_{l=1}^L \vec{a}^0_{i}\vec{x} + b^0_{i} + \Gamma \textstyle{\sum}_{l=1}^L |b^l_{i}| \leq 0
\label{box_coeff}
\end{equation}
which is a linear constraint.\\
On the other hand, if $\mat{A}$ is also uncertain, then equation \eqref{box_absolute} is equivalent to the following set of linear constraints
\begin{equation}
\begin{aligned}
\Gamma \textstyle{\sum}_{l=1}^L w^l_{i} + \vec{a}^0_{i}\vec{x} + b^0_{i} &\leq 0\\
- b^l_{i} - \vec{a}^l_{i}\vec{x} &\leq w^l_{i} &&\forall l \in 1,...,L\\
b^l_{i} + \vec{a}^l_{i}\vec{x} &\leq w^l_{i} &&\forall l \in 1,...,L\\
\end{aligned}
\label{box_linear}
\end{equation}

\subsection{Elliptical Uncertainty Sets} \label{ellip}
If the perturbation set $\mathcal{Z}$ is now an elliptical uncertainty set, i.e. $\textstyle{\sum}_{l=1}^L\frac{\zeta_l}{\sigma_l} \leq \Gamma$, then 
\begin{itemize}
	\item $\mat{F}\vec{\zeta} = [\mat{\sigma}^{-1}\vec{\zeta};\ 0]$ with $\mat{\sigma} = \text{diag}(\sigma_1,...,\sigma_L)$
	\item $\mat{G} = \mat{0}$, $\vec{h} = [\vec{0}_{L \times 1};\ \Gamma]$
	\item $\mathbf{K} = \left\{(\vec{z};t) \in \mathbb{R^L} \times \mathbb{R}: t > \|\vec{z}\|_{2} \right\}$
	\item The dual cone $\mathbf{K}^*$ = $\mathbf{K}$ (Lorentz or second order cone)
\end{itemize}
and therefore, equation \eqref{linear_counterparts} is equivalent to
\begin{equation}
\Gamma \sqrt{\textstyle{\sum}_{l=1}^L \sigma_l^2(- b^l_{i} - \vec{a}^l_{i}\vec{x})^2} + \vec{a}^0_{i}\vec{x} + b^0_{i} \leq 0
\label{ell_absolute}
\end{equation}
which is a second order conic constraint.\\
If only $\vec{b}$ is uncertain, i.e. $\mat{A}^l = 0 \quad \forall l = 1,2,...,L$, then equation \eqref{ell_absolute} will become
\begin{equation}
\vec{a}^0_{i}\vec{x} + b^0_{i} + \Gamma \sqrt{\textstyle{\sum}_{l=1}^L \sigma_l^2(b^l_{i})^2} \leq 0
\label{ell_coeff}
\end{equation}
which is a linear constraint ($\sqrt{\textstyle{\sum}_{l=1}^L \sigma_l^2(b^l_{i})^2}$ is a constant).
\begin{comment}
\subsection{Norm-one Uncertainty Sets}
Briefly, if the perturbation set is a norm-one uncertainty set, i.e. $\|\vec{\zeta}\|_1 \leq \Gamma$, then if $\mat{A}^l = 0$, the robust counterparts is equivalent to
\begin{equation}
\textstyle{\sum}_{l=1}^L \vec{a}^0_{i}\vec{x} + b^0_{i} + \Gamma \max_{l=1,..,L} |b^l_{i}| \leq 0
\label{rom_coeff}
\end{equation}
while if $\mat{A}$ is uncertain, then \eqref{linear_counterparts} is equivalent to
\begin{equation}
\begin{aligned}
\Gamma w_{i} + \vec{a}^0_{i}\vec{x} + b^0_{i} &\leq 0\\
- b^l_{i} - \vec{a}^l_{i}\vec{x} &\leq w_{i} &&\forall l \in 1,...,L\\
b^l_{i} + \vec{a}^l_{i}\vec{x} &\leq w_{i} &&\forall l \in 1,...,L\\
\end{aligned}
\label{rom_linear}
\end{equation}
\end{comment}

\section{Cones} \label{cones}
\subsection{Euclidean Space}
An Euclidean space is a finite dimensional linear space over real numbers equipped with an inner product $\langle x,y \rangle_E$.

\subsection{Cones}
A nonempty subset $\textbf{K}$ of an Euclidean space is called a cone if for any $x \in \textbf{K}$ and $\alpha \geq 0$ $\alpha x \in \textbf{K}$ \cite{bertsimas_tsitsiklis_1997}.\\
A cone is said to be convex cone if $\alpha, \beta \geq 0$ and $x, y \in \textbf{K}$, we have $\alpha x + \beta y \in \textbf{K}$ \cite{ben-tal_ghaoui_nemirovski_2009}.

\subsection{Dual Cones}
If $\textbf{K}$ is a cone in an euclidean space $\textbf{E}$, then the set 
\begin{equation*}
    \textbf{K}^* = \left\{ e \in \textbf{E}:\langle e,h \rangle_E \geq 0 \quad \forall h \in \textbf{K} \right\}
\end{equation*}
is also a cone and is called the cone dual to $\textbf{K}$ \cite{ben-tal_ghaoui_nemirovski_2009}.

\section{Equivalence Relations} \label{EqRel}
Let $\mathcal{R}$ be some relation on a set $\mathbf{S}$, and let $x,y \in \mathbf{S}$. We say $x \simeq y$ if $x$ and $y$ are related.
 
A relation $\mathcal{R}$ on a set $\mathbf{S}$ is called an equivalence relation if the following is true
\begin{itemize}
	\item the relation is reflexive, i.e. for all $a \in \mathbf{S}$, $a \simeq a$
	\item the relation is symmetric, i.e. for $a,b \in \mathbf{S}$, if $a \simeq b$, then $b \simeq a$
	\item the relation is transitive, i.e. for $a, b, c \in \mathbf{S}$, if $a \simeq b$ and $b \simeq c$, then $a \simeq c$
\end{itemize}
An equivalence relation naturally partitions a set into equivalence classes. Those classes are such that if $a$ and $b$ belong to the same class then $a \simeq b$, and if $a$ and $b$ are not related, then they belong to different classes.

\section{Signomial Programming} \label{sigProg}
Signomials allow us to solve non log-convex optimization problem as sequential geometric programs \cite{MARANAS1997351}. A signomial program has the following form
\begin{equation}
\begin{aligned}
&\text{minimize } &&f_{0}(\mathbf{x})\\
&\text{subject to } &&f_{i}(\mathbf{x}) &&\leq 1 \quad &i = 1, ...., m_p\\
& &&g_{i}(\mathbf{x}) -  h_{i}(\mathbf{x}) &&\leq 1 \quad &i = 1, ...., m_s
\end{aligned}
\label{sp}
\end{equation}
where $f_{i}$, $g_{i}$, and $h_{i}$ are posynomials.

To clarify how a signomial program is useful in our discussion, consider equation \eqref{linearCon_linPerts}, and assume that $\mathcal{Z}$ is an elliptical uncertainty set. Using the knowledge from subsection \ref{ellip}, \eqref{linearCon_linPerts} is equivalent to
\begin{equation}
\begin{aligned}
&\textstyle{\sum}_{k \in S_{i,j}}g_{i,j}^k v_{i,j}^k + s_{i,j}^{0.5} && \leq e^{t_{ij}} \qquad && \forall (i, j) \in \mathbf{P}\\
&\textstyle{\sum}_{l=1}^L \sigma_l s_{i,j}^{l} && \leq s_{i,j} \qquad && \forall (i, j) \in \mathbf{P}\\
&\left(\textstyle{\sum}_{k \in S_{i,j}}f_{i,j}^{k,l}v_{i,j}^k\right)^2 && \leq s_{i,j}^{l} \qquad \forall l \in 1,...,L \qquad && \forall (i, j) \in \mathbf{P}
\end{aligned}
\label{sp_compatible_constraints}
\end{equation}
If one of the `$f$'s is negative, then some constraints from the third set might not be GP-compatible, but SP-compatible. The robust geometric program will be a signomial program.
\begin{comment}
\end{comment}